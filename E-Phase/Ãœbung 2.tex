\documentclass[11pt, a4paper0]{scrartcl}
\usepackage[utf8]{inputenc}
\usepackage[T1]{fontenc}
\usepackage{mathptmx}
\usepackage[ngerman]{babel}
\usepackage{scrlayer-scrpage}
\usepackage{graphicx}
\usepackage{array}
\usepackage[right]{eurosym}
\usepackage[dvipsnames]{color}
\usepackage{tikz}
\usepackage{pgfplots}
\usepackage{gensymb}
\usepackage{amsmath,amssymb}
\usepackage[abs]{overpic}
\usepackage{todonotes}
\clubpenalty = 10000			%Keine Schusterjungen
\widowpenalty = 10000			%Keine Hurenkinder


%% einige KOMA-Einstellungen
\rohead{
	\begin{overpic}
		[width=0.2\textwidth]{header.png}%%% kleiner blauer Kasten oben rechts
		\hspace{-3cm}
		\put(150,5){\pagemark}
	\end{overpic}
}
\rehead{
	\begin{overpic}
		[width=0.2\textwidth]{header.png}
			\hspace{-3cm}
		\put(150,5){\pagemark}
	\end{overpic}
}
\cefoot{} %unterdrückt Seitenzahlen
\cofoot{}

\addtokomafont{pagehead}{\textnormal}
\renewcommand{\pnumfont}{\color{white}}


%%%% Zwischenüberschriften, Aufgabenbalken
\newcounter{numberOfExercise}

\newcommand{\Abschnitt}[1]{\stepcounter{numberOfExercise}
\hrule \vspace{3pt}
\noindent \textbf{Aufgabe~\arabic{numberOfExercise}} $\left(\textnormal{#1}\right)$
\vspace{1pt}\hrule width 1\textwidth height 0.5pt
\ \\[0.1cm]
} % Für Aufgabenstellungen mit Text vor dem itemize

\newcommand{\AbschnittV}[1]{\stepcounter{numberOfExercise}
	\hrule \vspace{3pt}
	\noindent \textbf{Aufgabe~\arabic{numberOfExercise}} $\left(\textnormal{#1}\right)$
	\vspace{1pt}\hrule width 1\textwidth height 0.5pt
} % Für Aufgaben ohne Text, itemize direkt nach Balken
%%%% Überschrift, Kopf
\newcommand{\Kopfzeile}[4]{
	\noindent \huge{\textbf{#1}}\\
	\hrule \vspace{0.5ex}
		\noindent \Large{\textbf{#2 . Übungsblatt}} \hfill #4\\
		\noindent \large{\textbf{#3}}\\
% 1: Mathe-Nachhilfe 2: Übungsblatt-Nr. 3: Beschreibung 4: Datum
	\vspace{-1.2ex}	\hrule width 1\textwidth height 0.5pt
	\ \\

}
%%%% Geometrie des Blattes
	\usepackage[left=2cm,right=1.5cm,include headfoot,top=1.5cm,bottom=2cm]{geometry}
	
%%%%%%%%%%%%%
%Befehle und Kommandos
%%%%%%%%%%%%%
\newcommand{\sinp}[1]{\sin \left( #1 \right)}
\newcommand{\cosp}[1]{\cos \left( #1 \right)}
\newcommand{\tanp}[1]{\tan \left( #1 \right)}
\newcommand{\dx}{\,\mathrm{d}x}
\newcommand{\dy}{\,\mathrm{d}y}

%Einheiten
\newcommand{\m}{\mathrm{m}}
\newcommand{\cm}{\mathrm{cm}}
\newcommand{\LE}{\mathrm{LE}}
\newcommand{\FE}{\mathrm{FE}}
\newcommand{\const}{\mathrm{const.}}
\begin{document}
\Kopfzeile{Mathematik\\Einführungsphase}{2}{Kurvendiskussion}{}
\Abschnitt{Ableitung}
	Bilden Sie jeweils die erste und zweite Ableitung.
	\begin{alignat*}{3}
		&f(x)=3x^4-7x^3+x^2+5x-9 \qquad && g(x)=2x-\sqrt{x} && h(x)=sin(x)-3x\\
		&i(x)=2x^{\frac{3}{5}}-\frac{3}{2}x^{\frac{2}{3}}+3 && j(x)=\frac{1}{2\sqrt{x}} && k(x)=nx^{3n}\\
		&l(x)=3x^{-2}+4x^{-1}+x && m(x)=\frac{3}{4}x+\frac{7}{8}x-\frac{3}{2}x^3+\frac{5}{6}x^3 \qquad && n(x)=4x+x^2-2x^4
	\end{alignat*}
\AbschnittV{Kurvendiskussion und Extremalproblem}
	\begin{itemize}
		\item[a)] Gegeben ist die ganzrationale Funktion~$f$ mit~$f(x)=0,5x^4-3x^2+4 \quad x \in \mathbb{R}$
	\begin{itemize}
		\item[1.] Untersuchen Sie die Funktion~$f$ auf Symmetrie und Schnittpunkte mit den Koordinatenachsen.
		\item[2.] Ermitteln Sie Extrem- und Wendepunkte des Graphen der Funktion~$f$.
		\item[3.] Stellen Sie den Graphen von~$f$ im Intervall $-2,5\le x \le2,5$ dar.
		\item[4.] Zwischen dem Graphen von~$f$ und der~$x$-Achse soll ein Rechteck mit maximalem Flächeninhalt einbeschrieben werden. Ermitteln Sie die Länge der Rechteckseiten und die Größe des maximalen Flächeninhalt.
	\end{itemize}

		\item[b)] Gegeben ist die Funktion~$f$ mit~$f(x)=2+2*cos(x)$.
	\begin{itemize}
		\item[1.] Untersuchen Sie den Graphen auf Symmetrie.
		\item[2.] Bestimmen Sie die Nullstellen von~$f$ im Intervall~$\left[0;2\pi\right]$ und den Schnittpunkt des Graphen mit der~$y$-Achse.
		\item[3.]Ermitteln Sie Extrem- und Wendepunkte des Graphen im Intervall~$[0;2\pi]$
		\item[4.]Skizzieren Sie den Verlauf von~$f$ im Intervall~$\left[0;2\pi\right]$ in ein geeignetes Koordinatensystem.
		\item[5.]Zwischen~$0$ und~$\pi$ wird in die Fläche zwischen der Kurve und der~$x$-Achse ein Rechteck so einbeschrieben, dass die Koordinatenachsen eine Begrenzung und die Koordinaten eines Punktes~$P\left(r|s\right)$ auf~$f$ die andere Begrenzung bilden. Bestimmen Sie die Lage von~$P$ so, dass der Umfang des Rechtecks extremal wird. Geben Sie den Umfang jeweils an.
	\end{itemize}
	\item[c)] Eine Zündholzschachtel soll~$5\mathrm{cm}$ lang sein und das Volumen~$45\mathrm{cm}^3$ haben. Bei welcher Breite und Höhe braucht man zur Herstellung am wenigsten Material?
	\end{itemize}
\newpage

\Abschnitt{Modellierung von ganzrationalen Funktionen}
%Abitur: 2012, A2/ 2010, A2/ Erfolg im Mathe-Abi, S.11
		Eine kleine Firma stellt Mountainbikes her. Bei einer Monatsproduktion von x Mountainbikes entstehen Fixkosten in 	Höhe von 5000 Euro und variable Kosten V(x) (in Euro), die durch folgende Tabelle modellhaft gegeben sind:\\
		\renewcommand{\arraystretch}{1.2}
	\begin{center}
		\begin{tabular}{|c|c|c|c|c|}\hline
			x	&	0	&	2	&	6	&	10\\	\hline		%Wird die Tabelle direkt unterhalb der Aufgaben-
			V(x)&	0	&	306	&	954	&	1650\\	\hline		%stellung angezeigt?!
		\end{tabular}
	\end{center}
		\renewcommand{\arraystretch}{1}
	\begin{itemize}
		\item[a)]Bestimmen Sie die Funktionsgleichung der ganzrationalen Funktion 2. Grades~$V(x)$ sowie der monatlichen Herstellungskosten~$H$ in Abhängigkeit von~$x$.\\
			Skizzieren Sie den Graph von~$H$ für~$0 \leqslant x \leqslant 200$ in ein geeignetes Koordinatensystem.\\
			Bei welcher Produktionszahl sind die variablen Kosten fünfmal so hoch wie die Fixkosten?\\
		\item[b)]Alle monatlich produzierten Mountainbikes werden zu einem Preis von~$450~\textnormal{Euro}$ pro Stück an einen Händler verkauft.\\
			Geben Sie den monatlichen Gewinn~$G$ in Abhängigkeit von~$x$ an und skizzieren Sie den Graphen der Gewinnfunktion in das vorhandene Koordinatensystem.\\
			Bei welchen Produktionszahlen macht die Firma Gewinn?\\
			Wie hoch ist der maximale Gewinn pro Monat?\\
		\item[c)]Durch große Konkurrenz auf dem Markt muss die Firma den Preis pro Mountainbike senken.\\
			Um wie viel Prozent vom ursprünglich erzielten Preis ist dies höchstens möglich, wenn pro Monat~$90$ Mountainbikes produziert werden und der Gewinn mindestens~$2000\textnormal{Euro}$ betragen soll?\\[0,3cm]
	\end{itemize}
\Abschnitt{Anwendungsbeispiel}%Idee: aktuelle Norm herausfinden! Bilder!
	Für Skisprungschanzen, auf denen offizielle Wettkämpfe stattfinden, hat der internationale Skiverband (FIS) Normen erstellt. In einer älteren Norm steht:\\
	"Die Anlaufbahn besteht aus einem möglichst geradlinig gestalteten Teil mit der Neigung~$\gamma$, einem anschließenden kreisförmigen Übergangsbogen mit dem Radius~$r$ und dem geradlinig verlaufenden Schanzentisch mit der Neigung~$\alpha$ und der Länge~$t$."\\[0,2cm]
	Die Mühlenkopfschanze in Willingen wurde nach ihrem Umbau 2001 von der FIS als offizielle Wettkampfstätte anerkannt. Von der Schanze sind folgende Angaben bekannt:\\
	In einem Koordinatensystem (alle Angaben in Meter) beginnt der Anlauf im Punkt~$P_A(0;50,64)$. Der obere geradlinige Teil hat die Länge~$56,3m$ und die Neigung~$\gamma=35\degree$ gegenüber der Horizontalen. Der kreisförmige Übergangsbogen hat den Mittelpunkt~$M(106,36;104,35)$, der Schanzentisch beginnt im Punkt~$P_S(86,32;1,28)$, hat die Länge~$6,7m$ und die Neigung~$\alpha=11\degree$ gegenüber der Horizontalen. Der gesamte Anlauf, also auch der Schanzentisch, ist nach unten geneigt.
	\begin{itemize}
		\item[a)]Skizzieren Sie den Verlauf des Anlaufs in einem Koordinatensystem.
		\item[b)]Bestimmen Sie die Geradengleichung für den geradlinigen Anlauf und die Koordinaten des Punktes~$P_E(x_E;y_E)$, an dem dieser endet. [\emph{zur Kontrolle:}~$P_E(46,12;18,35)$(auf 2 Nachkommastellen gerundet)]
		\item[c)]Berechnen Sie~$m_1$ und~$m_2$ gemäß der Definition im untenstehenden Kasten und anschließend das Produkt~$m_1 \cdot m_2$.\\
		\emph{Bonus:} Erklären Sie die Bedeutung des letzten Ergebnisses im Sachzusammenhang.\\[-0,3cm]
		\renewcommand{\arraystretch}{1.2}
		\begin{center}
			\begin{tabular}{|cl|} \hline
				(1) &	$m_1$= Steigung des oberen geradlinigen Anlaufteils\\
				(2)	&	$m_2$= Steigung der Geraden durch~$M$ und~$P_E$\\ \hline
			\end{tabular}
		\end{center}
		\renewcommand{\arraystretch}{1}
		\item[d)]In einer neueren FIS-Norm steht, dass die gekrümmte Übergangskurve eine kubische Parabel, d.~h. eine Parabel dritter Ordnung, sein darf.\\
		Geben Sie vier Bedingungen an, damit sich die kubische Parabel~$p_3$ mit
		\begin{equation*}
			p_3(x)=-0,000028053x^3+0,011864312x^2-1,61556349x+70,3757036
		\end{equation*}
		in guter Näherung in den Punkten~$P_E$ und~$P_S$ ohne Knick und ohne Sprung an die geradlinigen Teile des Anlaufs und des Schanzentisches anschmiegt. Prüfen Sie, ob die Bedingungen für den Punkt~$P_E$ erfüllt sind.\\[0,3cm]
		\emph{Hinweis:} Alle vier Bedingungen sind bereits im Text enthalten und müssen nur noch als Gleichungen aufgeschrieben werden.
	\end{itemize}
\end{document}