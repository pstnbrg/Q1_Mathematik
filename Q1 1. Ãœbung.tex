\documentclass[11pt, a4paper0]{scrartcl}
\usepackage[utf8]{inputenc}
\usepackage[T1]{fontenc}
\usepackage{mathptmx}
\usepackage[ngerman]{babel}
\usepackage{scrlayer-scrpage}
\usepackage{graphicx}
\usepackage{array}
\usepackage[right]{eurosym}
\usepackage[dvipsnames]{color}
\usepackage{tikz}
\usepackage{pgfplots}
\usepackage{gensymb}
\usepackage{amsmath,amssymb}
\usepackage[abs]{overpic}
\usepackage{todonotes}
\clubpenalty = 10000			%Keine Schusterjungen
\widowpenalty = 10000			%Keine Hurenkinder


%% einige KOMA-Einstellungen
\rohead{
	\begin{overpic}
		[width=0.2\textwidth]{header.png}%%% kleiner blauer Kasten oben rechts
		\hspace{-3cm}
		\put(150,5){\pagemark}
	\end{overpic}
}
\rehead{
	\begin{overpic}
		[width=0.2\textwidth]{header.png}
			\hspace{-3cm}
		\put(150,5){\pagemark}
	\end{overpic}
}
\cefoot{} %unterdrückt Seitenzahlen
\cofoot{}

\addtokomafont{pagehead}{\textnormal}
\renewcommand{\pnumfont}{\color{white}}


%%%% Zwischenüberschriften, Aufgabenbalken
\newcounter{numberOfExercise}

\newcommand{\Abschnitt}[1]{\stepcounter{numberOfExercise}
\hrule \vspace{3pt}
\noindent \textbf{Aufgabe~\arabic{numberOfExercise}} $\left(\textnormal{#1}\right)$
\vspace{1pt}\hrule width 1\textwidth height 0.5pt
\ \\[0.1cm]
} % Für Aufgabenstellungen mit Text vor dem itemize

\newcommand{\AbschnittV}[1]{\stepcounter{numberOfExercise}
	\hrule \vspace{3pt}
	\noindent \textbf{Aufgabe~\arabic{numberOfExercise}} $\left(\textnormal{#1}\right)$
	\vspace{1pt}\hrule width 1\textwidth height 0.5pt
} % Für Aufgaben ohne Text, itemize direkt nach Balken
%%%% Überschrift, Kopf
\newcommand{\Kopfzeile}[4]{
	\noindent \huge{\textbf{#1}}\\
	\hrule \vspace{0.5ex}
		\noindent \Large{\textbf{#2 . Übungsblatt}} \hfill #4\\
		\noindent \large{\textbf{#3}}\\
% 1: Mathe-Nachhilfe 2: Übungsblatt-Nr. 3: Beschreibung 4: Datum
	\vspace{-1.2ex}	\hrule width 1\textwidth height 0.5pt
	\ \\

}
%%%% Geometrie des Blattes
	\usepackage[left=2cm,right=1.5cm,include headfoot,top=1.5cm,bottom=2cm]{geometry}
	
%%%%%%%%%%%%%
%Befehle und Kommandos
%%%%%%%%%%%%%
\newcommand{\sinp}[1]{\sin \left( #1 \right)}
\newcommand{\cosp}[1]{\cos \left( #1 \right)}
\newcommand{\tanp}[1]{\tan \left( #1 \right)}
\newcommand{\dx}{\,\mathrm{d}x}
\newcommand{\dy}{\,\mathrm{d}y}

%Einheiten
\newcommand{\m}{\mathrm{m}}
\newcommand{\cm}{\mathrm{cm}}
\newcommand{\LE}{\mathrm{LE}}
\newcommand{\FE}{\mathrm{FE}}
\newcommand{\const}{\mathrm{const.}}
\begin{document}
\Kopfzeile{Mathematik\\Qualifikationsphase 1}{1}{Einführung in die Integralrechnung}{}
\Abschnitt{Fläche unter Funktionen}
Ermitteln Sie die Flächen unter den Funktionen in Abb.~$1$ sowie Abb.~$2$.
\begin{figure}[h!]
	\begin{center}
		\def\svgwidth{0.3\textwidth}
		\input{1_1_gleichfoermige_Bewegung.pdf_tex}
		\caption{gleichförmige Bewegung}
	\end{center}
\end{figure}

\todo[inline, color=white]{To Do: Grafiken einfügen}
\AbschnittV{Ober- und Untersumme/ Streifenmethode}
\begin{itemize}
\item[a)] Gegeben sei die Funktion~$f(x)=x^2$. Zerlegen Sie das Intervall~$0 \leq x \leq 3$
	\begin{itemize}
		\item[1.] in 3 Streifen
		\item[2.] in 6 Streifen.
	\end{itemize}
Geben Sie jeweils die Ober- und Untersummen an.
\item[b)] Sei wie in Aufgabenteil a)~$f(x)=x^2$ und~$I=\left[0,3\right]$.
	\begin{itemize}
		\item[1.] Leiten Sie ausgehend von der Funktion~$f(x)$ folgende Gleichung für die Obersumme her.
		\begin{equation*}
			O_n = \left(\frac{x}{n}\right)^3 \cdot \frac{1}{6} n \left(n + 1\right)\left(2n + 1\right)
		\end{equation*}
		\item[2.] Bestimmen Sie allgemein die Obersumme~$O_n$ bis zur Stelle~$x=3$ in Abhängigkeit von der Streifenbreite~$n$.
		\item[3.] Bestimmen Sie nun die Obersumme~$O_n$ bis zur Stelle~$x$ in Abhängigkeit von~$x$ und~$n$.
		\item[4.] Bilden Sie den Grenzwert~$\lim\limits_{n \to \infty}$.
		\item[5.] Bilden Sie die erste Ableitung des Grenzwertes aus Aufgabenteil 4. Was fällt Ihnen auf?
	\end{itemize}
\end{itemize}
\AbschnittV{Stammfunktion}
\begin{itemize}
	\item[a)] Verbinden Sie jede Funktion~$f$ mit möglichen Stammfunktionen~$F$.
	
	\begin{alignat*}{2}
		&						 \qquad \qquad \qquad && F(x)= \frac{1}{3} x^8\\
		&f(x)= x^2				 \qquad \qquad \qquad && F(x)= \frac{1}{3} x^3\\
		&						 \qquad \qquad \qquad && F(x)= \frac{1}{24} x^8\\
		&f(x)= \frac{1}{3} x^7	 \qquad \qquad \qquad && F(x)= x\\
		&						 \qquad \qquad \qquad && F(x)= \frac{1}{3} x^3 + 5\\
		&f(x)= 11				 \qquad \qquad \qquad && F(x)= \frac{1}{24} x^8 + 8\\
		&						 \qquad \qquad \qquad && F(x)= e^x\\
		&f(x)= e^x				 \qquad \qquad \qquad && F(x)= 2 e^x + \frac{3}{4}\\
		&						 \qquad \qquad \qquad && F(x)= \frac{7}{3} x^6
	\end{alignat*}
	\item[b)] Geben Sie zu den in Teilaufgabe a) noch nicht zugeordneten Stammfunktionen~$F$ einen Funktionsterm für~$f$ an, sodass~$F$ eine Stammfunktion von~$f$ ist.
	\end{itemize}
\AbschnittV{Erste Integrationsregeln}
\begin{itemize}
	\item[a)] Bestimmen Sie die folgenden unbestimmten Integrale mit Hilfe der Faktorregel.
	\begin{alignat*}{3}
		& \int 3x^2 \dx \qquad \qquad && \int 4 x^3 \dx \qquad \qquad	&& \int \frac{2}{x} \dx \\
		& \int 3 e^x \dx && \int \frac{1}{3} \sinp{x} \dx && \int \frac{4}{y^2} \dy
	\end{alignat*}
	\item[b)] Bestimmen Sie die folgenden unbestimmten Integrale mit Hilfe der Summenregel.
	\begin{alignat*}{3}
		& \int x^2 + 3 x^3 \dx \qquad \qquad && \int \frac{1}{82} x^{16} + x^7 \dx \qquad \qquad && \int \frac{1}{x} + \ln \left(x\right) \dx\\
		& \int 5 x^2 + x - 8 \dx && \int \sinp{x} + \cosp{x} \dx && \int y + 8y^2 + 7 y^4 \dy\\
	\end{alignat*}
\end{itemize}
\AbschnittV{Hauptsatz der Differential- und Integralrechnung}
\begin{itemize}
	\item[a)] Berechnen Sie mit Hilfe des Hauptsatzes der Differential- und Integralrechnung folgende bestimmte Integrale.
	
	\centering
	\fbox{\parbox{0.3\linewidth}{$\int_{a}^{b} f = \left[F(x)\right]_a^b = F(b) - F(a)$}}
	
	\begin{alignat*}{3}
		& \int_{1}^{6} \left(\frac{1}{2} x^3 + 2x\right) \dx \qquad \qquad && \int_{-1}^{1} (x^6 + 3x^5 - 2x^4) \dx  \qquad \qquad && \int_{-2}^{3} x \dx\\
		& \int_{3}^{5} (3x+1) \dx && \int_{-3}^{-1} -2(x+2)^2 \dx && \int_{2}^{5} \xi^2 \mathrm{d \xi}\\
		& \int_{0}^{\pi} \sinp{x} \dx && \int_{-\pi}^{\pi} \cosp{x} \dx && \int_{-1}^{1} \frac{\dx}{x}
	\end{alignat*}
\end{itemize}
\AbschnittV{Weiterführende Aufgaben}
\begin{itemize}
	\item[a)] Die Funktion~$f$ ist abschnittsweise definiert. Berechnen Sie das angegebene Integral.
		\begin{equation*}
			\int_{0}^{10} f \dx \quad \textnormal{mit} \quad f(x)=\left\{\renewcommand{\arraystretch}{1.2}\begin{array}{ll}
			\frac{1}{2} x+1 & \textnormal{für}~~ 0 \leq x \leq 2\\
			2				& \textnormal{für}~~ 2 \leq x \leq 4\\
			\frac{1}{2}		& \textnormal{für}~~ 4 \leq x\\
			\end{array}\right.
		\end{equation*}
		\begin{equation*}
			\int_{0}^{5} f \dx \quad \textnormal{mit} \quad f(x)=\left\{
			\begin{array}{ll}
			2x+1	& \textnormal{für}~~ 0\leq x \leq 2\\
			x+3		& \textnormal{für}~~ 2\leq x \leq 5
			\end{array}\right.\
		\end{equation*}
		\begin{equation*}
			\int_{1}^{6} f \dx \quad \textnormal{mit} \quad f(X)=\left\{
			\begin{array}{ll}
			x-1	& \textnormal{für}~~ 1 \leq x \leq 2\\
			1	& \textnormal{für}~~ 2 \leq x \leq 4\\
			2x-7& \textnormal{für}~~ 4 \leq x
			\end{array}\right.
		\end{equation*}
	\item[b)] Sei~$g:\left[-3,1\right] \to \mathbb{R}$ mit
	\begin{equation*}
		g(x)=\left\{
		\begin{array}{ll}
		x^3 + 6x^2 + 12x + 8 & \textnormal{für}~~ -3\leq x \leq -1\\
		x^2 -\frac{1}{3} & \textnormal{für}~~ -1\leq x \leq 1
		\end{array}\right.\
	\end{equation*}
	abschnittsweise definiert. Berechnen Sie~$\int_{-3}^{1}g(x)dx$ und interpretieren Sie das Ergebnis.
		\renewcommand{\arraystretch}{1.0}
\end{itemize}
\todo[inline, color=white]{Anwendungsbeispiel aus der Physik/ Mechanik}
\AbschnittV{Anwendungsbeispiele des Integrals}
\begin{itemize}
	\item[a)] In der Physik spielt der magnetische Fluss~$\Phi$, der durch eine Leiterschlaufe führt, eine große Rolle.
	\begin{equation*}
		\Phi = \int B \cdot \cosp{\varphi}\cdot \mathrm{dA}
	\end{equation*}
	
	\item[b)] In der technischen Mechanik werden Belastungen, die auf Bauteile mit bestimmten Stoffeigenschaften und Geometrien wirken, vor der Konstruktion berechnet. Gegeben sei ein einseitig eingespannter Balken der Länge~$3l$ mit der Torsionssteifigkeit~$GI_T$, die in diesem Beispiel als konstant angenommen werden kann. Auf den Balken wirkt der Torsionsmomentenverlauf~$M_{T1}=m_0x+2m_0 l$ bei~$0\leq x \leq 2l$ sowie~$M_{T2}=4m_0 l$ bei~$2l \leq x \leq 3l$. Ferner gilt für den Verdrehwinkel~$\vartheta$:
	\begin{equation*}
		\vartheta ' = \frac{M_T}{GI_T}
	\end{equation*}
\end{itemize}
\end{document}